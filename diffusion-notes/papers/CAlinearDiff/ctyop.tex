\documentclass[12pt,a5paper]{article}
\usepackage{amsmath,natbib,defns,reducecode}
\IfFileExists{ajr.sty}{\usepackage{ajr}}{}
\title{holistic discretisation that ensures continuity between 
adjacent elements}
\author{AJR}
\date{Feb 2002 -- \today}

\begin{document}

\maketitle

Execute in Reduce with \verb|in_tex "ctyop.tex"$|

Seeks to model the 1D advection-diffusion \pde
\begin{equation*}
\D tu=-c\D xu+\DD xu
\end{equation*}
on a macroscale grid, for `small' advection speed~\(c\).
The \(j\)th element is \(X_{j-1}\leq x\leq X_j\).

Improve printing.
\begin{reduce}
on div; off allfac; on revpri;
factor hh,uu,c,d;
\end{reduce}

Define shift right/left operators \verb|ep| and~\verb|em|: use that in terms of centred mean and difference operators, \(\mu\)~and~\(\delta\), they are \(1\pm\mu\delta+\rat12\delta^2\) \cite[p.65]{npl61}.
Also encode the identity that \(\mu^2=1+\delta^2/4\)\,.
Define the `spline' operator \(\verb|ss|=S:=(1+\delta^2/6)^{-1}\).
\begin{reduce}
ep:=1+mu*del+del^2/2;
em:=1-mu*del+del^2/2;
let { mu^2=>1+del^2/4
    , ss*del^2=>6-6*ss };
\end{reduce}

Write the solution in terms of the microscale variable~\(\xi:=(x-X_{j-1})/H\).
\begin{reduce}
depend xi,x; 
let df(~a,x)=>df(a,xi)/hh;
\end{reduce}

To find corrections, linear operator~\verb|linv| solves \textsc{de}s of the form \(\DD \xi{\hat u}=\text{Res}\) such that \(\hat u=0\) at \(\xi=0,1\).
\begin{reduce}
operator linv; linear linv;
let { linv(xi^~~p,xi)=>(xi^(p+2)-xi)/(p+1)/(p+2)
    , linv(1,xi)=>(xi^2-xi)/2 };
\end{reduce}

Write the slow manifold in terms of amplitudes \(U_j(t):=u(X_j,t)\).  
These depend upon time according to \(\de t{U_j}=g_j\)\,.
We let all the \(j\)~dependence be in the operators.
\begin{reduce}
depend uu,t; 
let df(uu,t)=>g;
\end{reduce}

The linear solution are equilibria, \(g=0\), of piecewise linear field between \(U_{j-1}\) at \(\xi=0\) and \(U_j\) at \(\xi=1\).
\begin{reduce}
g:=0;
u:=xi*uu+(1-xi)*em*uu;
\end{reduce}

Iterate until the slow manifold model is found to the following specified order of accuracy.
Resolving to errors~\Ord{c^3} in the advection speed~\(c\) allows us to explore any stabilising effect of our analysis in the presence of otherwise destabilising advection.
\begin{reduce}
let { gamma^7=>0, c=>0 };
for it:=1:99 do begin 
\end{reduce}

Compute residuals of governing equations.
\begin{reduce}
    pde:= -df(u,t)+df(u,x,x)-c*df(u,x);
    amp:=sub(xi=1,u)-uu;
    cty:=sub(xi=0,ep*u)-sub(xi=1,u);
    hux:=hh*df(u,x)$
    jmp:=-sub(xi=0,ep*hux)+sub(xi=1,hux)
        +(1-gamma)*sub(xi=1,ep*u-2*u+em*u);
    write lengthres:=map(length(~a),{pde,amp,cty,jmp});
\end{reduce}

Correct approximations based upon the residuals.
These ad hoc corrections are not optimal, but they do work after enough iterations.
\begin{reduce}
    g:=g+(gd:=-ss*jmp/hh^2);
    u:=u-linv(pde-(xi+(1-xi)*em)*gd,xi)*hh^2;
\end{reduce}

Exit the loop when all residuals are zero to the order specified.
\begin{reduce}
    showtime;
    if {pde,amp,cty,jmp}={0,0,0,0} then write it:=it+10000;
end;
\end{reduce}


Get equivalent \pde, but need to improve to be able to analyse to any order.
\begin{reduce}
ssd:=1-hh^2*d^2/6+hh^4*d^4/72-hh^6*d^6/2160;
let d^7=>0;
gde:=sub(ss=ssd,g);
\end{reduce}

Explore the sawtooth mode for which \(\delta=i2\), \(\mu=0\) and \(S=1/(1+\delta^2/6)=3\)\,.
Appears to be very good convergence to the correct value of~\(-\pi^2/H^2\): roughly a significant digit accuracy for each order in~\(\gamma\).
\begin{reduce}
gsaw:=sub({ss=3,mu=0,del=i*2},g*hh^2/uu);
on rounded; print_precision 6$
gsawsumonpi2:=gsaw*(for n:=0:10 sum gamma^n)/pi^2;
off rounded;
\end{reduce}

This appears to simplify the form of the evolution:
introducing \(\verb|gamdel2|:=\gamma\delta^2\).
\begin{reduce}
factor gamdel2;
g:=(g where ss*gamma=>gamma-ss/6*gamdel2);
u:=(u where { ss*gamma=>gamma-ss/6*gamdel2
            , gamma*del^2=>gamdel2})$
\end{reduce}

Fin.
\begin{reduce}
end;
\end{reduce}

\bibliographystyle{agsm}
\bibliography{bibexport,bib}

\end{document}
