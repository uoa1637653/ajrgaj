\documentclass[11pt,a5paper]{article}
\IfFileExists{ajr.sty}{\usepackage{defns,ajr}}{}
\title{Two intervals with Dirichlet BCs}
\author{AJR}
\date{\today}

\usepackage{reducecode}

\begin{document}

\maketitle

Let's try the simplest problem.  
The finite domain is \(-1\leq x\leq1\), \(t\)~is time, and we seek a field~\(u(x,t)\) satisfying the heat\slash Burgers' \pde\ \(u_t=u_{xx}-\alpha uu_x\) and Dirichlet physical boundary conditions of \(u=0\) at \(x=\pm1\)\,.
The simplest numerical approximation in our class is formed by imposing the artificial breakpoint at \(x=0\) that the field is continuous, \([u]_0=0\), but the derivative has a jump of \([u_x]_0=(1-\gamma)(0-2u|_0+0)=-2(1-\gamma)u|_0\).

\begin{reduce}
on div; off allfac; on revpri;
factor uu,alpha;
\end{reduce}
The slow manifold depends upon \(\verb|uu|=U(t):=u(0,t)\).
Its evolution is \(dU/dt=\verb|g|\).
\begin{reduce}
depend uu,t;
let df(uu,t)=>g;
\end{reduce}
Slow subspace approximation to the slow manifold.
\begin{reduce}
u:=(1-xx)*uu;
g:=0;
\end{reduce}
The subgrid field is expressed in terms of~\(\verb|xx|:=|x|\) so define \(\verb|sx|:=\text{sign}\,x\):
\begin{reduce}
depend xx,x; 
let { df(xx,x)=>sx, sx^2=>1 };
\end{reduce}

\begin{reduce}
operator iint;linear iint;
let { iint(1,x)=>(xx^2-xx)/2
    , iint(xx^~~p,x)=> (xx^(p+2)-xx)/(p+1)/(p+2) };
\end{reduce}

\begin{reduce}
let { gamma^3=>0 };
alpha:=alfa*gamma;
aa:=0;% 1/6 may be useful for low orders, but only a little at high
for it:=1:19 do begin
\end{reduce}

Compute the residual of the \pde\ and coupling condition.
\begin{reduce}
write
respde:=-df(u,t)+df(u,x,x)-alpha*u*df(u,x);
ux:=df(u,x);
write
rescc:=sub(xx=0,(1-aa*gamma)*(sub(sx=1,ux)-sub(sx=-1,ux))
    +2*(1-gamma)*u);
\end{reduce}
Compute corrections from residuals.
\begin{reduce}
g:=g+(gd:=3/2*rescc+3*sub({xx=1,sx=0},int(respde*(1-xx),xx)));
u:=u+iint(gd*(1-xx)-respde,x);
\end{reduce}

Terminate loop when residuals are zero.
\begin{reduce}
if {respde,rescc}={0,0} then write "success ",it:=10000+it;
end;
\end{reduce}


\begin{reduce}
end;
\end{reduce}


\end{document}
