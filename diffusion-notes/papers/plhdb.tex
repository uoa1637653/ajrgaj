\documentclass[12pt,a4paper]{article}
\usepackage{graphicx,subcaption}
\usepackage{amsmath,amsfonts,defns}
\usepackage{pgfplots}

\newcommand{\Bb}[1]{%
  \expandafter\def\csname#1#1\endcsname%
  {\ensuremath{\mathbb #1}}}
\Bb X\Bb T\Bb R\Bb I\Bb J
\newcommand{\secref}[1]{\ref{sec:#1}}

\title{An application of piecewise-linear holistic discretisation}
\author{G.A. Jarrad and A.J. Roberts}

\begin{document}
\maketitle

\begin{abstract}
The fidelity of numerical simulation of a spatio-temporal dynamical system is 
laregely constrained by the chosen discretisation of the PDE. 
In particular, the modeller is typically free to choose
arbitrary finite-element forms for the various spatial derivatives,
without necessarily having knowledge of the accuracy of the resulting numerical schemes.
The holistic discretisation approach described in this paper obviates the problem of arbitration. % mediation?

Centre manifold theory is applied to derive an asymptotically accurate
representation of the microscale dynamics of the one-dimensional Burgers' equation. 
In the process, the corresponding macroscale dynamics 
are constrained to match the microscale solution at discrete grid-points.
The resulting representation of macroscale evolution provides an unambiguous discretisation of the PDE,
suitable for numerical simulation. 

The iterative process starts with a choice of the leading, macroscale approximation to the linearised system.
Suitable internal boundary conditions are then induced on the spatial derivatives 
at the end-points of each discrete interval.
Although the choice of IBCs might appear to be arbitrary, they are in fact governed by the placement of the grid-points and the 
form of the leading discrete approximation.
The particular approach taken here is to start with a piecewise linear but continuous approximation,
in contrast to similar analyses that use piecewise constant, discontinuous approximations.
This is motivated by the principle that a more accurate leading approximation should lead to faster
convergence of the asymtoptic solution, via the Rayleigh-Ritz theorem.

Further iterations of the centre manifold process lead inexorably to a temporal evolution formulation of the macroscale dynamics,
holistically informed by the underlying microscale dynamics.  
We examine the accuracy and stability of the resulting numerical scheme, in comparison to
the behaviours of several other typical approximations.
\end{abstract}

\numberwithin{equation}{section}
\numberwithin{figure}{section}
\numberwithin{table}{section}
\section{Introduction}\label{sec:intro}
As an application of holistic discretisation, consider the numerical simulation of
an arbitrary solution $u:\XX\times \TT\mapsto \RR$ to the advection--diffusion (Burgers') equation
\begin{eqnarray}
	\D tu = \nu\DD xu - \alpha u\D xu\,.
\label{eq:burgers}
\end{eqnarray}
%Mention cole-hopf leads to known solution which is unconditionally stable.
The first step is to spatially discretise $\XX$ into a sequence of $L$ intervals bounded by $N=L+1$  grid-points,
 ${\vec X}=[X_j]_{j=0}^{L}$, 
whereupon the continuum dynamics $u$ are summarised by the coarse dynamics
${\vec U}=[U_j]_{j=0}^{L}$, for $U_j(t):=u(X_j,t)$ for all $t\in \TT$.
For convenience, it may be supposed henceforth that the grid spacing is regular,
namely $X_{j+1}-X_j=H$ for all $j=0,1,\ldots,L-1$.

The next step is to specify the explicit form of the coarse temporal evolution
\begin{eqnarray}
	\dot{\vec U}(t) = {\vec g}({\vec U}(t))\,.
\label{eq:temporal}
\end{eqnarray}
The traditional approach is to use centred approximations, e.g.\ $\delta^2 U_j/H^2$ for $\DD xu$
and $U_j\mu\delta U_j/H$ for $u\D xu$, 
where $\delta=\sigma^{\frac{1}{2}}-\sigma^{-\frac{1}{2}}$,
$\mu=(\sigma^{\frac{1}{2}}+\sigma^{-\frac{1}{2}})/2$,
and $\sigma U_j=U_{j+1}$.
However, the advection term has another plausible representation, namely the
conservative form $\mu\delta U_j^2/2H$.
For illustrative purposes, a mixture of the two latter representations will be used for comparison,
namely
\begin{eqnarray}
	\dot{U}_j = \nu\frac{\delta^2 U_j}{H^2}-(1-\theta)\alpha\frac{U_j\mu\delta U_j}{H}
-\theta\alpha\frac{\mu\delta U_j^2}{2H}\,.
\label{eq:mixture}
\end{eqnarray}
In contrast, the holistic approach has no such representational ambiguity, and, 
as shown in Section~\secref{nonlin}, gives rise at first-order to the model
\begin{eqnarray}
	\dot{U}_j = S\left(\nu\frac{\delta^2 U_j}{H^2}-\alpha\frac{U_j\mu\delta U_j}{3H}
-\alpha\frac{\mu\delta U_j^2}{3H}\right)\,,
\label{eq:holistic1}
\end{eqnarray}
where $S=(1+\delta^2/6)^{-1}$. Observe that, apart from the non-local corrective operator $S$,
this holistic model matches the mixture model~\eqref{eq:mixture} for $\theta=\frac{2}{3}$.
This parameter value is exactly the critical value predicted by Fornberg \cite{} 
to be necessary for stable simulation (with $\nu=0$ and $\alpha=1$) for a selection of numerical integration schemes.
Further comparisons of the numerical behaviours of the holistic and mixture models are given in Section~\secref{numeric}.

\section{Motivation}\label{sec:motive}
The choice of a piecewise linear leading approximation, in contrast to the usual piecewise constant one
(citations), is motivated by an application of the Rayleigh--Ritz theorem.
Consider a general dynamical system of the form
\begin{eqnarray}
\dot{\vec u} = {\cal L}{\vec u}+{\cal N}({\vec u})\,,
\label{eq:nonlin:gen}
\end{eqnarray}
where it is pre-supposed for convenience that ${\cal N}({\vec 0})={\vec 0}$. 
The slow manifold approximation to the dynamics of the linearised system
$\dot{{\vec u}}={\cal L}{\vec u}$ about ${\vec u}={\vec 0}$ is then given by
\begin{eqnarray}
{\vec u} = V{\vec s}\,, && \dot{{\vec s}} = \Lambda{\vec s}\,,
\end{eqnarray}
whereupon $V\Lambda{\vec s}={\cal L}V{\vec s}$ for arbitrary ${\vec s}$.
Observe now that a particular choice of $\Lambda=\mbox{diag}(\left[\lambda_j\right])$ uncouples this relation into the components
${\cal L}{\vec v}_j=\lambda{\vec v}_j$, where $V=\left[{\vec v}_j\right]$.

Hence, under the strong assumption that ${\cal L}$ is self-adjoint, the Rayleigh--Ritz theorem indicates that
$\lambda_j = R({\vec v}_j)$ for the Rayleigh quotient 
\[ R({\vec v}) = \frac{\langle{\vec v},{\cal L}{\vec v}\rangle}{\|{\vec v}\|^2}\,.\]
Now, upon perturbing ${\vec v}_j$, e.g.\ choosing ${\vec u}={\vec v}_j+\epsilon{\vec w}$, it can be shown
that $R({\vec u})=\lambda_j+O(\epsilon)$ for arbitrary ${\vec w}$, and further that
 $R({\vec u})=\lambda_j+O(\epsilon^2)$ in the special case that ${\vec w}\perp{\vec v}_j$.
This provides support for the general expansions
\begin{eqnarray}
{\vec u} \sim V{\vec s}+\epsilon{\vec w}_1+\epsilon^2{\vec w}_2+\ldots\,,
&& 
\dot{\vec u} \sim V\Lambda{\vec s}+\epsilon{\vec g}_1+\epsilon^2{\vec g}_2+\ldots\,,
\end{eqnarray}
for the nonlinear dynamical system~\eqref{eq:nonlin:gen}.

\section{Linear analysis}\label{sec:lin}
The linearisation of equation~\eqref{eq:burgers} takes the form $\dot{\vec u}={\cal L}{\vec u}$ for the linear operator
${\cal L}=\nu\DD x{}$. It can be shown that ${\cal L}$ is self-adjoint under any of these boundary conditions:
periodic ($U_L=U_0$); Dirichlet ($U_0=U_L=0$); or Neuman ($U'_0=U'_L=0$).

***
Indeed, any piecewise constant solution is an equilibrium, leading to its frequent use as an initial approximation.
[cite]
Such an approximation is discontinuous in u and hence is typically coupled with a continuity condition on $u_x$,
subject to a homotopic smoothing parameter, gamma.
In contrast, the approach taken here is to start with a continuous, piecewise linear approximation 
(which is an equilibrium of the linearised system obtained with alpha=0) 
and to induce a suitable condition on $u_x$.

\section{Nonlinear analysis}\label{sec:nonlin}

\section{Numerical stability}\label{sec:numeric}

\section{Conclusion}

\end{document}
